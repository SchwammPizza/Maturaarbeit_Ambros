\section{Methode}
\subsection{Allgemeines Rechnen}
Um Bilder vom Buddhabrot zu generieren wurde mit der Programmiersprache Julia eine 2-Dimensionale Matrix erstellt, bei dem jeder Wert der Matrix zu einem Pixel zugehört. Zuerst muss man jedoch wissen, welche Punkte man iterieren muss, so wird zuerst das Mandelbrot ausgerechnet. Um Speicherplatz zu sparen, ordnet man ihnen die Werte 0 und 1 zu: 1 divergiert gegen $\infty$, 0 gehört der Mandelbrotmenge an. Alle Punkte, die gegen $\infty$ divergieren, werden nochmals iteriert, nun wird geschaut wo die Punkte durchgehen. Am Ende wird geschaut, welcher Punkt die meisten Treffer bekam, den dieser Wert wird gebraucht, um die Grauabstufung zu machen. Wie man sicher schon merkt, muss mal 3 Mal die riesige Matrix durchgehen und auch 2-Mal iterieren. Dies ist somit im Vergleich zur Mandelbrotmenge sehr rechenaufwändig, denn bei der wäre nur ein Durchlauf nötig. Als Deffinitionsbereich wird $\{z \in \mathbb{C}\text{ }|\text{ }-2 < \Re(z) < 1 \& -1 < \Im(z) < 1\}$ gewählt.
\subsection{Erstellen eines ersten Zoom}
Beim ersten Ansatz wurde für den Zoom eine riesige Matrix, welche der Zoom im Quadrat so gross ist, wie das zu erwartene Bild, erstellt mit all den Werten drin, die man braucht. Dann wurde der Teil ausgewählt, den man haben möchte und lässt den Zeichnen. Dies mithilfe der Berechnung vom Offset im Array, welches Bewertstelligt wird, indessen man angibt welches der zum Bezoomende Punkt ist.
\subsection{CUDA optimierung}
Um lange Wartezeiten zu vermeiden, wurde der Code mit CUDA formuliert. Es wurde CUDA.jl hinzugefügt, um so mehrere Punkte gleichzeitig rechnen zu lassen. Dadurch, dass nun auf der Grafikkarte gerechnet wird, hat man weniger Speicherplatz zur Verfügung. Für dies müssen die Funktionen umgeschrieben werden, damit CUDA dies kennt. Zu bemerken ist, dass CUDA eine Plattform um auf der Grafikkarte zu Rechnen ist, welche durch CUDA.jl mit Julia genutz werden kann. CUDA ist von Nvidia, was dazu führt, dass das Programm nurnoch auf PC laufen kann, die eine Grafikkarte von Nvidia haben.
\subsection{Analyse}
Da das Buddhabrot als chaotisches System gilt, versucht man im Chaos ein Muster zu finden. Als erstes wurde geschaut, ob ein gegebener Bereich einen überwiegender Einfluss hat auf einen Bereich, als die anderen Bereichen oder auch markant keinen Einfluss. So könnte man in einem Zoom, nur noch diesen Bereich anschauen oder eben nicht. Dies wurde einfach bewertstelligt, indessen man Bilder erstellte, die zeigten, wie sich Punkte aus diesen Bereichen iterierten. Zuerst wurden die 4 Quadranten als Startbereiche gewählt.
\\
Als zweites wurde noch zusätzlich die Überlegung gemacht, dass der absolute Wert von $c$ ebenfalls einen Einfluss haben könnte, wie wenn der kleiner als 1 wäre. Dies wurde getestet indessen man eine Variable dem vorigem Aufbau mitgab, welches mit einem XOR dafür sorgte, dass entweder der Bereich, bei dem $|c|>=1$ gilt, oder der andere ausgewertet wird.
\\
Die daraus Resultierende Bilder wurden Analysiert, indessen man die Flächen bestimmte, welche Pixelwerte haben von RGB\{Float64\}(0,0,0) mit sehr kleiner Tolleranz.
\subsection{Implementierung der Ergebnisse}
Der zu verwerfende Bereich wurde schon in der Mandelbrotberechnung verworfen, um Zeit zu sparen. Die anderen nützlichen Ergebnissen (In Kapitel Ergebnisse schauen) wurde geschaut in welchen Bereichen auf den Analysenbilder es Schwarz ist oder nur ein Treffer gezeigt wird. Danach wurde diese Fläche zu einer Funktion umgewandelt. Mit dieser wird geschaut, ob der massgebende Eckpunkt des Zoomsbereich in der Fläche ist. Falls es so ist, wird dessen nicht Quadranten miteinbezogen. So muss man jenachdem nurnoch 2 Quadranten von einem Bereich berechnen.