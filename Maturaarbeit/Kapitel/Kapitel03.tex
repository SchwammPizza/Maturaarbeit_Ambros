\section{Methode}
\subsection{Allgemeines Rechnen}
Um Bilder vom Buddhabrot zu generieren wurde mit der Programmiersprache Julia eine 2-Dimensionale Matrix erstellt, bei dem jeder Wert der Matrix zu einem Pixel zugehört. Zuerst muss man jedoch wissen, welche Punkte man iterieren muss, so rechnet man zuerst das Mandelbrot aus. Um Speicherplatz zu sparen, ordnet man ihnen die Werte 0 und 1 zu: 1 gehört nicht der Mandelbrotmenge an, 0 gehört der Mandelbrotmenge an. Alle Punkte, die ins Unendliche divergieren, werden nochmals iteriert, nun wird geschaut wo die Punkte durchgehen. Am Ende wird geschaut, welcher Punkt die meisten Treffer bekam, den dieser Wert braucht man um die Grauabstufung zu machen. Wie man sicher schon merkt, muss mal 3 Mal die riesige Matrix durchgehen und auch 2-Mal iterieren. Dies ist somit im Vergleich zur Mandelbrotmenge sehr rechenaufwändig bei der nur ein Durchlauf nötig wäre.
\subsection{Erstellen eines ersten Zoom}
Beim ersten Ansatz wurde für den Zoom eine riesige Matrix, welche der Zoom im Quadrat so gross ist wie das erwartene Bild, erstellt mit all den Werten drin, die man braucht. Dann wurde der Teil ausgewählt, den man haben möchte und lässt den Zeichnen. Ein 16-Facher Zoom ist nicht mehr möglich, da die Matrix so gross wird, dass es nicht genügend RAM freiräumen kann. Ebenfalls dauert es dann schnell mal viel Zeit, ein 12-facher Zoom, mit der Auflösung 4'001 auf 2'667 Pixel, zum Punk -1.25, dauerte etwa 45.6h. So wurde der Code folgender weise optimiert: Es wurde CUDA.jl hinzugefügt, um so mehrere Punkte gleichzeitig rechnen zu lassen. Dadurch, dass nun auf der Grafikkarte gerechnet wird, hat man weniger Speicherplatz zur Verfügung. Man kann nur noch einen 6.5-fachen Zoom machen. Dies ist eigentlich egal, denn es geht ja darum, die Rechenleistung zu verringern und einen Allfälliger Allgorythmus zu finden. Das gleiche Programm, ein 1-facher Zoom, dauert nun anstelle von 3 Stunden nurnoch 2 Minuten.
\subsection{Analyse}
Da die Buddhabrotmenge als chaotisches System gilt, versucht man im Chaos ein Muster zu finden. Als erstes wurde geschaut, ob ein gegebener Bereich einen überwiegender Einfluss hat auf einen Bereich, als die anderen Bereichen oder auch markant keinen Einfluss. So könnte man in einem Zoom, nur noch diesen Bereich anschauen oder eben nicht. Dies wurde einfach bewertstelligt, indessen man Bilder erstellte, die zeigten, wie sich Punkte aus diesen Bereichen iterierten. Zuerst wurden die 4 Quadranten als Startbereiche gewählt. Man merkt nun, dass die Bereiche eindeutigen Einfluss erheben.
\\
Als zweites wurde noch zusätzlich die Überlegung gemacht, dass der absolute Wert von $c$ ebenfalls einen Einfluss haben könnte, wie wenn der kleiner als 1 wäre. Dies wurde getestet und es zeigte sich das vom Bereich $\{z \in \mathbb{C}\text{ }|\text{ } |z| >1\text{ } \& \text{ } 1>=\text{Re}(z)>=0 \text{ } \& \text{ }-1<=\text{Im}(z)<=1\}$ nur ein maximaler Treffer von 4 erreicht wird. Somit kann dieser Bereich vernachlässigt werden in der Berechnung, da 4 von in der Norm 69 Maximum sehr wenig ist und vor allem der Beinflusste Bereich nicht wirklich erkenntlich ist beim Anblick des Buddhabrotes.
\subsection{Analyseergebnisse ins Programm Implementieren}
Dazu wurde geschaut in welchen Bereichen auf den Analysenbilder es Schwarz ist oder nur ein Treffer gezeigt wird. Danach wurde diese Fläche zu einer Funktion umgewandelt. Mit dieser wird geschaut, ob der massgebende Eckpunkt des Zoomsbereich in der Fläche ist. 