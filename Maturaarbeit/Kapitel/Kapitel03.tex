\section{Methode}
\subsection{Allgemeines Rechnen}
Um Bilder vom Buddhabrot zu generieren, wurde mit der Programmiersprache Julia eine zweidimensionale Matrix erstellt, bei der jeder Wert der Matrix zu einem Pixel zugeordnet wurde. Zuerst mussten die Punkte, welche man iterieren sollte, jedoch bekannt sein. So wurde zuerst das Mandelbrot ausgerechnet. Um Speicherplatz zu sparen, ordnete man ihnen die Werte 0 und 1 zu: 1 divergiert gegen $\infty$, 0 gehört der Mandelbrotmenge an. Alle Punkte, die gegen $\infty$ divergieren, wurden nochmals iteriert. Anschliessend wurde geschaut, wo die Punkte durchgingen. Am Ende wurde ermittelt, welcher Punkt die meisten Treffer bekam, denn dieser Wert war nötig, um die Grauabstufung zu machen. Daher musste man dreimal die riesige Matrix durchgehen und auch zweimal iterieren. Dies war somit im Vergleich zur Mandelbrotmenge sehr rechenaufwändig, bei der nur ein Durchlauf nötig gewesen wäre. Als Definitionsbereich wurde $\{z \in \mathbb{C}\text{ }|\text{ }-2 < \Re(z) < 1 \& -1 < \Im(z) < 1\}$ gewählt.
\subsection{Erstellen eines ersten Zooms}
Beim ersten Ansatz wurde für den Zoom eine riesige Matrix erstellt, in die anschliessend mit all den darin enthaltenen nötigen Werten gezoomt wurde. Die Grösse der Matrix war die Zoomtiefe im Quadrat, wie das vom erwarteten Bild. Dann wurde ein Ausschnitt gewählt und liess diesen zeichnen. Dies mithilfe der Berechnung vom Offset im Array, welcher durch die Angabe, zu welchem Punkt man zoomen möchte, berechnet wurde.
\subsection{CUDA-Optimierung}
Um lange Wartezeiten zu vermeiden, wurde der Code mit CUDA formuliert. Es wurde CUDA.jl hinzugefügt, um so mehrere Punkte gleichzeitig rechnen zu lassen. Dadurch, dass nun der grösste Teil des Codes auf der Grafikkarte (GPU) gerechnet wurde, war weniger Speicherplatz für die Berechnung verfügbar. Drei Funktionen wurden hierbei auf der GPU gerechnet: die Berechnung der Punkte die nach $\infty$ divergierten, die Berechnung des Buddhabrotes und das Zeichnen des Ausschnitts. Für dies mussten die Funktionen umgeschrieben werden, damit CUDA diese nutzen konnte. Zu erwähnen ist, dass CUDA eine Plattform ist um auf der Grafikkarte zu rechnen. CUDA wird in Julia durch CUDA.jl genutzt. CUDA ist von Nvidia, was dazu führt, dass das Programm nur noch auf Computern laufen kann, die eine Grafikkarte von Nvidia haben.
\subsection{Analyse}
Da das Buddhabrot als chaotisches System gilt, wurde versucht, im Chaos ein Muster zu finden. Eingangs wurde geschaut, ob ein gegebener Bereich gegenüber anderen Bereichen einen überwiegenden Einfluss ausübt oder sehr geringen Einfluss hat. So könnte man in einem Zoom nur noch diesen Bereich anschauen oder vernachlässigen. Dies wurde einfach bewerkstelligt, indem man Bilder erstellte, die zeigten, wie sich Punkte aus diesen Bereichen iterierten. Zuerst wurden die 4 Quadranten als Startbereiche gewählt.
\\
Anschliessend wurde noch zusätzlich die Überlegung gemacht, dass der absolute Wert von $c$ ebenfalls einen Einfluss haben könnte, wie wenn er kleiner als 1 wäre. Dies wurde getestet, indem man eine Variable dem vorigen Aufbau mitgab, welcher mit einem XOR dafür sorgte, dass entweder der Bereich, bei dem $|c|>=1$ gilt, oder der andere ausgewertet wurde.
\subsection{Implementierung der Ergebnisse}
Um Zeit zu sparen, wurde der zu verwerfende Bereich schon in der Mandelbrotberechnung verworfen. Bei den anderen nützlichen Ergebnissen (vgl. Kapitel Ergebnisse) wurde geschaut, in welchen Bereichen es auf den Analysebildern schwarz war oder nur ein Treffer gezeigt wurde. Danach wurde diese Fläche zu einer Funktion umgewandelt. Mit dieser wurde geschaut, ob der massgebende Eckpunkt des Zoombereichs in der Fläche war. Wenn es so war, wurde deren Quadrant nicht miteinbezogen. So musste man je nachdem nur noch 2 Quadranten für ein Bild berechnen.