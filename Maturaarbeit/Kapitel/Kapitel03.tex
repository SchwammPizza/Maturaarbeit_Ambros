\section{Methode}
\subsection{Allgemeines Rechnen}
Um Bilder vom Buddhabrot zu generieren, wurde mit der Programmiersprache Julia eine 2-dimensionale Matrix erstellt, bei der jeder Wert der Matrix zu einem Pixel zugeordnet wird. Zuerst muss man jedoch wissen, welche Punkte man iterieren muss, so wird zuerst das Mandelbrot ausgerechnet. Um Speicherplatz zu sparen, ordnet man ihnen die Werte 0 und 1 zu: 1 divergiert gegen $\infty$, 0 gehört der Mandelbrotmenge an. Alle Punkte, die gegen $\infty$ divergieren, werden nochmals iteriert. Dann wird geschaut, wo die Punkte durchgehen. Am Ende wird geschaut, welcher Punkt die meisten Treffer bekam, denn dieser Wert wird gebraucht, um die Grauabstufung zu machen. Wie man sicher schon merkt, muss man dreimal die riesige Matrix durchgehen und auch zweimal iterieren. Dies ist somit im Vergleich zur Mandelbrotmenge sehr rechenaufwändig, dabei wäre nur ein Durchlauf nötig. Als Definitionsbereich wird $\{z \in \mathbb{C}\text{ }|\text{ }-2 < \Re(z) < 1 \& -1 < \Im(z) < 1\}$ gewählt.
\subsection{Erstellen eines ersten Zooms}
Beim ersten Ansatz wurde für den Zoom eine riesige Matrix erstellt, in die anschliessend gezoomt wird mit all den Werten drin, die man braucht. Die Grösse der Matrix ist die Zoomtiefe im Quadrat, wie das vom erwarteten Bild. Dann wurde der Ausschnitt ausgewählt, den man haben möchte und lässt diesen Zeichnen. Dies mithilfe der Berechnung vom Offset im Array, welcher durch die Angabe, zu welchem Punkt man zoomen möchte, berechnet wird.
\subsection{CUDA-Optimierung}
Um lange Wartezeiten zu vermeiden, wurde der Code mit CUDA formuliert. Es wurde CUDA.jl hinzugefügt, um so mehrere Punkte gleichzeitig rechnen zu lassen. Dadurch, dass nun auf der Grafikkarte gerechnet wird, hat man weniger Speicherplatz zur Verfügung. Für dies müssen die Funktionen umgeschrieben werden, damit CUDA diese kennt. Zu bemerken ist, dass CUDA eine Plattform ist, um auf der Grafikkarte zu rechnen. CUDA wird in Julia durch CUDA.jl genutzt. CUDA ist von Nvidia, was dazu führt, dass das Programm nur noch auf Computern laufen kann, die eine Grafikkarte von Nvidia haben.
\subsection{Analyse}
Da das Buddhabrot als chaotisches System gilt, versucht man im Chaos ein Muster zu finden. Als erstes wurde geschaut, ob ein gegebener Bereich einen überwiegenden Einfluss auf einen Bereich ausübt, als die anderen Bereichen oder auch markant keinen Einfluss. So könnte man in einem Zoom nur noch diesen Bereich anschauen oder vernachlässigen. Dies wurde einfach bewerkstelligt, indem man Bilder erstellte, die zeigten, wie sich Punkte aus diesen Bereichen iterierten. Zuerst wurden die 4 Quadranten als Startbereiche gewählt.
\\
Als zweites wurde noch zusätzlich die Überlegung gemacht, dass der absolute Wert von $c$ ebenfalls einen Einfluss haben könnte, wie wenn er kleiner als 1 wäre. Dies wurde getestet, indem man eine Variable dem vorigen Aufbau mitgab, welcher mit einem XOR dafür sorgte, dass entweder der Bereich, bei dem $|c|>=1$ gilt, oder der andere ausgewertet wird.
\subsection{Implementierung der Ergebnisse}
Der zu verwerfende Bereich wird schon in der Mandelbrotberechnung verworfen, um Zeit zu sparen. Bei den anderen nützlichen Ergebnissen (In Kapitel Ergebnisse schauen) wird geschaut, in welchen Bereichen auf den Analysenbildern es schwarz ist oder nur ein Treffer gezeigt wird. Danach wird diese Fläche zu einer Funktion umgewandelt. Mit dieser wird geschaut, ob der massgebende Eckpunkt des Zoombereichs in der Fläche ist. Falls es so ist, wird deren Quadrant nicht miteinbezogen. So muss man je nachdem nur noch 2 Quadranten für ein Bild berechnen.