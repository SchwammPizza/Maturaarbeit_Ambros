\section*{Vorwort}
Das Unendliche zieht die Menschen schon eh und je an, so auch mich. Es ist unbeschreiblich, doch spielen und rechnen wir damit gerne. Dazu kommt, dass unsere Technik immer besser und leistungsfähiger wird. Dies spielt sich zu, denn desto mehr Leistung man hat, umso schneller und näher kann man sich der Unendlichkeit approximieren. Ich bin schon seit ich klein bin, von komplexeren mathematischen Vorgängen beeindruckt, deshalb schaue ich in meiner Freizeit gerne YouTube-Videos über solche. Eines Tages sah ich das Buddhabrot auf dem Titelbild eines Videofilms. Als ich es mir dann anschaute, verstand ich nichts davon, denn es war auf Englisch. So beschäftigte ich mich vorerst nur mit der Mandelbrot-Menge. Erst als ich in der Schule die Fraktalen kennenlernte, verstand ich es besser. So entschloss ich mich nun, das Verstehen des Buddhabrotes nochmals zu versuchen und mich mit dem Buddhabrot auseinander zu setzen. So schreibe ich unter anderem deshalb eine Arbeit darüber.\\
\\
Ich möchte mich bei einigen Personen bedanken. Zuerst bei meiner Betreungsperson Fabio Thöny, der mir bei jeglichen Fragen zur Verfügung stand und mich stets unterstützte. Ebenfalls ist meiner Mutter, meiner Schwester und dem Freund meiner Schwester zu danken, die meine Arbeit entgegenlasen und korrigierten. Auch Julian Steiner danke ich, der mir beim Programmieren mental zur Seite stand und mir bei allfälligen Problemen wie auch bei den Formulierungen in der Arbeitgeholfen hat. Ebenfalls geht mein Dank an Linus Romer, der mir bei der CUDA-Implementierung half.