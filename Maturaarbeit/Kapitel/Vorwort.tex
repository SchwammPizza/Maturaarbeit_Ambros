\section*{Vorwort}
Das Unendliche zieht die Menschen schon seit eh und je an, so auch mich. Die Unendlichkeit ist unbeschreiblich, doch spielen und rechnen wir gerne mit ihr. Da die Technologie immer besser und leistungsfähiger wird, ist es möglich, sich der Unendlichkeit immer schneller und näher zu approximieren. Schon seit meiner Jugend beeindrucken mich komplexere mathematische Vorgänge, weshalb ich mir in meiner Freizeit gerne YouTube-Videos über solche ansehe. Eines Tages entdeckte ich das Buddhabrot auf dem Titelbild eines Video. Als ich mir dieses dann anschaute, verstand ich aufgrund der englischen Sprache nichts davon. So beschäftigte ich mich vorerst nur mit der Mandelbrot-Menge. Erst als ich in der Schule die Fraktale kennenlernte, verstand ich die Thematik etwas besser. So entschloss ich mich, das Verstehen des Buddhabrotes nochmals zu versuchen und mich mit dem diesem auseinanderzusetzen. Aus diesem Grund befasse ich mich in dieser Arbeit mit dieser Thematik.\\
\\
Vorab möchte ich einigen Personen meinen Dank aussprechen. In erster Linie bei meiner Betreuungsperson Fabio Thöny, der mir bei jeglichen Fragen zur Verfügung stand und mich stets unterstützte. Ebenfalls ist meiner Mutter, meiner Schwester und dem Freund meiner Schwester zu danken, die meine Arbeit gegenlasen und korrigierten. Auch Julian Steiner danke ich, der mir beim Programmieren mental zur Seite stand und mir bei allfälligen Problemen wie auch bei den Formulierungen in der Arbeit geholfen hat. Ebenfalls geht mein Dank an Linus Romer, der mir bei der CUDA-Implementierung half.\\
\\
Falls Sie diese Arbeit und die aus ihr resultierenden Bilder samt Codes interessieren, besuchen Sie den unten angefügten Link:\\ 
\url{https://github.com/SchwammPizza/Maturaarbeit_Ambros}