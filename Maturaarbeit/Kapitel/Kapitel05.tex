\section{Disskussion}
Bei den Analysen der Quadranten hat sich gezeigt, dass die Quadranten auf Bereichen kein Einfluss geben und auf gewissen Starken. Die Quadranten hatten auf Folgende Bereiche keinen Einfluss im Definitionsbereich:
\begin{center}
\subsubsection*{1. Quadrant}
$\{z \in \mathbb{C}\text{ }|\text{ }\Im(z)>57.4\frac{\Re(z)+2}{11.4}-\frac{29.8}{3.8}\}$
\subsubsection*{2. Quadrant}
$\{z \in \mathbb{C}\text{ }|\text{ }((\Re(z)-\frac{3'504}{667})^2+\Im(z)^2>6.25 \text{ }\&\text{ } \Im(z) < -\frac{25}{667}) \lor (\Im(z)>(\Re(z)-1)^2) \}$
\subsubsection*{3. Quadrant}
$\{z \in \mathbb{C}\text{ }|\text{ }((\Re(z)-\frac{3'504}{667})^2+\Im(z)^2>6.25 \text{ }\&\text{ } \Im(z) > \frac{25}{667}) \lor (\Im(z)<-(\Re(z)-1)^2) \}$
\subsubsection*{4. Quadrant}
$\{z \in \mathbb{C}\text{ }|\text{ }\Im(z)<-57.4\frac{\Re(z)+2}{12}+\frac{24.5}{4}\}$\\
\end{center}
Bei den durchgeführten Analysen mit den Radien hat sich gezeigt, dass vom Bereich $\{z \in \mathbb{C}\text{ }|\text{ } |z| >1\text{ } \& \text{ } 1>=\Re(z)>=0 \text{ } \& \text{ }-1<=\Im(z)<=1\}$ nur ein maximaler Treffer von 4 erreicht wird. Somit kann dieser Bereich vernachlässigt werden in der Berechnung, da 4 von in der Norm 69 Maximum sehr wenig ist und vor allem der Beinflusste Bereich nicht wirklich erkenntlich ist beim Anblick des Buddhabrotes.