\section{Diskussion}
Bei den Analysen der Quadranten hat sich gezeigt, dass die Quadranten auf bestimmte Bereiche keinen, hingegen auf gewisse Bereiche einen starken Einfluss ausüben. Die Quadranten hatten auf folgende Bereiche keinen Einfluss im Definitionsbereich:
\begin{center}
\subsubsection*{1. Quadrant}
$\{z \in \mathbb{C}\text{ }|\text{ }\Im(z)>57.4\frac{\Re(z)+2}{11.4}-\frac{29.8}{3.8}\}$
\subsubsection*{2. Quadrant}
$\{z \in \mathbb{C}\text{ }|\text{ }((\Re(z)-\frac{3'504}{667})^2+\Im(z)^2>6.25 \text{ }\&\text{ } \Im(z) < -\frac{25}{667}) \lor (\Im(z)>(\Re(z)-1)^2) \}$
\subsubsection*{3. Quadrant}
$\{z \in \mathbb{C}\text{ }|\text{ }((\Re(z)-\frac{3'504}{667})^2+\Im(z)^2>6.25 \text{ }\&\text{ } \Im(z) > \frac{25}{667}) \lor (\Im(z)<-(\Re(z)-1)^2) \}$
\subsubsection*{4. Quadrant}
$\{z \in \mathbb{C}\text{ }|\text{ }\Im(z)<-57.4\frac{\Re(z)+2}{12}+\frac{24.5}{4}\}$\\
\end{center}
Bei den durchgeführten Analysen mit den Radien hat sich gezeigt, dass vom Bereich $\{z \in \mathbb{C}\text{ }|\text{ } |z| >1\text{ } \& \text{ } 1>=\Re(z)>=0 \text{ } \& \text{ }-1<=\Im(z)<=1\}$ nur ein maximaler Treffer von 4 erreicht wird. Somit kann dieser Bereich in der Berechnung verworfen werden, da 4 in Relation zu den maximal erreichten Treffern von 69 vernachlässigbar ist, da es unter anderem beim Anblick des Buddhabrotes nichtmal gut erkenntlich ist.