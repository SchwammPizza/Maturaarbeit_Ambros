\section{Fazit}
Eine Steigerung vom ersten Ansatz zur letzten optimierten Fassung ist klar ersichtlich, abgesehen davon, dass das Bild nun am hellsten ist. Zwar wurde ein Punkt gewählt, bei dem sich die letzte Variante gelohnt hat, wobei auch die Punkte in dieser Umgebung sehr spannend sind. Hätte man einen anderen Punkt ausgewählt, wäre keine klare Verbesserung ersichtlich, womöglich denn sogar nicht mal vorhanden gewesen. Ebenfalls ist das letzte Programm nicht gleich effizient wie das zweite, da es nun mehr if-Konditionen hat, welche so das Programm verlangsamen.\\
Es gibteiniges, was man hätte probieren können, um einen tieferen Zoom zu ermöglichen. Beispielsweise die Auflösung des Bildes auf 1080p herunter zu schrauben. Heutzutage ist ein 4K-Bild jedoch üblicher und die Bilder werden heller, da mehr Punkte iteriert werden. Ein weiteres Beispiel wäre eine Datenbank, welche während des Rechnens erstellt würde, so dass die Threads bei alten Resultaten weiter rechnen könnten. Das Programm wäre zwar wiederum langsamer, da nun mehr Aufrufe ausserhalb der CUDA geschehen. Eine weitere Methode wäre, den Metropolis-Hashtings Algorithmus von Alexander Boswell zu nutzen. Dies wurde nicht gemacht, da die Wahrscheinlichkeit, nicht alle Punkte miteinzubeziehen, vorhanden ist. Hier wird nämlich die Wahrscheinlichkeit vom Divergieren eines Punktes berechnet. Dass alle Punkte miteinbezogen werden, ist durch den Verwerfungsbereich in der letzten Variante dieser Arbeit ebenfalls nicht gegeben, jedoch war der maximale Treffer 4 bei verschieden grossen Iterationsstufen vernachlässigbar. Ebenfalls sind hier die Punkte bestimmt nicht vorhanden und nicht von einer Wahrscheinlichkeit abhängig.