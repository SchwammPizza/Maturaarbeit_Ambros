\section{Fazit}
Eine Steigerung vom ersten Ansatz zur letzten optimierten Fassung ist klar ersichtlich, abgesehen davon, dass das Bild nun ab hellsten ist. Zwar wurde ein Punkt gewählt, bei dem es die letzte Variante gelohnt hat, jedoch sind auch die Punkte in dieser Umgebung sehr spannend. Hätte man einen anderen Punkt ausgewählt, wäre keine klare Verbesserung ersichtlich, wenn denn sogar nicht einmal vorhanden. Ebenfalls ist das letzte Programm nicht gleich effizient wie das zweite, da es nun mehr if-Konditionen hat, welche so das Programm verlangsamen.\\
Es gibt Einiges, was man hätte probieren können, um einen tieferen Zoom zu machen. Ein Beispiel wäre eine Datenbank, welche während des Rechnens erstellt würde, so dass die Threads bei alten Resultaten hätten weiter rechnen können. Das Programm wäre zwar wiederum langsamer, da nun mehr Aufrufe ausserhalb der CUDA geschehen. Eine andere Methode wäre den Metropolis-Hashtings Algorithmus von Alexander Boswell zu nutzen. Dies wurde nicht gemacht, da dort die Wahrscheinlichkeit, dass nicht alle Punkte miteinbezogen werden, vorhanden ist. Hier wird nämlich die Wahrscheinlichkeit vom Divergieren eines Punktes berechnet. Dass alle Punkte miteinbezogen werden, ist durch den Verwerfungsbereich in der letzten Variante dieser Arbeit ebenfalls nicht gegeben, jedoch war der maximale Treffer 4 bei verschieden grossen Iterationsstufen. Ebenfalls sind hier die Punkte bestimmt nicht vorhanden und nicht von einer Wahrscheinlichkeit abhängig.