\section{Präliminarien}
\subsection{Komplexe Zahlen}
Wenn man mit den reelen Zahlen arbeitet, bekommt man Probleme, wenn man die Wurzel aus einer negativen Zahl zieht. Jedoch haben Mathematiker im 16. Jahrhundert eine Lösung dafür entdeckt, indem wir unser Zahlensystem mit den imaginären Zahlen, die die imaginären Einheit i haben, erweitern. Addieren und Subtrahieren zweier imaginären Zahlen funktioniert genau gleich, wie wenn man mit einer Variabel rechnet. Dies heisst, dass man das i wie ein $x$ beim Formeln vereinfachen behandeln kann. Jedoch beim Multiplizieren, Dividieren und beim somit entstehenden Rechnen mit Potenzen muss man aufpassen, denn es gilt für $n \in \mathbb{Z}$:
% Gleichunugen
\begin{align*}
 \text{i}^{4n} &= 1 \\
\text{i}^{4n+1} &= \text{i} \\
\text{i}^{4n+2} &= -1 \\
\text{i}^{4n+3} &= -\text{i} 
\end{align*}
%Weiter im Text
Hat man einer der Fälle, ist dies dann zu ersetzen. Hier sieht man gut, dass man die Verknüpfung der imaginären Zahlen reel sein kann, wie es auch andersrum war. Wenn man nun eine imaginäre Zahl $\text{i}b$ mit einer reellen Zahl $a$ zusammenaddiert, bekommt man eine komplexe Zahl  $c=a+\text{i}b$ mit dem Realteil $a$ und dem Imaginärteil $\text{i}b$. $a$ und $b$ sind hier reelle Zahlen. Beim Addieren für die Komplexe Zahlen $z=a+\text{i}b$ und $w=e+\text{i}f$
\[z+w\]
\[a+\text{i}b+e+\text{i}f\]
\[a+e+(b+f)\text{i}\]
\\
Nun merkt man, dass diese Zahl mit einem Vektor vergleichbar ist, denn um die Zahl darstellen zu können benutzt man die 2-Dimensionale komplexe Ebene. Daraus schliesst sich, dass komplexe Zahlen 2-Dimensional sind. Schaut man dies in der komplexe Ebene an, fängt der Punkt an, sich scheinbar unkontrolliert herumspringen, folgt jedoch weiterhin logischen Regeln, beim Quadrieren verschiebt sich der Punkt in die positive Drehrichtung(Gegenuhrzeigersinn). Ebenfalls kann, da die Zahl vergleichbar mit einem Vektor ist, den absoluten Betrag der komplexen Zahl $c$ bestimmt werden, welcher mit dem Pythagoras berechnet wird:
\[|c| = |a+\text{i}b| = \sqrt[2]{a^2+b^2} \]

\subsection{Fraktale Geometrie}
\subsubsection{Allgemein}
Um zum Buddhabrot zu kommen, müssen wir noch einen weiteren Begriff klären, das Fraktal.\\ Würde bei einem 3$n$ grossem Strich der mittlere Teil fehlen, stattdessen der Rest eines gleichseitigen Dreiecks dort stehen, hätte man die erste Iteration einer Kochkurve. Würde man nun in die $n$ grosse Striche die gesamte Kochkurve einfügen, muss man dies ab nun immer wieder machen, sodass es schwer vorstellbar wird. Wenn man jedoch in die Kockurve reinzoomt, findet man die Kochkurve immer wieder: ein rekursives Bild oder eben ein Fraktal. Man definiert nun das Fraktal als eine Figur, bei der sehr oft Selbstähnlichkeit auffindbar ist, dass heisst, dass das gesamte Fraktal oder Teile davon mehrfach im Fraktal vorkommen und das Fraktal selbst eine gebrochene und somit keine ganzzahlige Dimension besitzt.\\
\\
In der Abbildung \ref{fig:Kochkurve} ist die Entstehung der Kochkurve zu sehen.

\begin{figure}[h]
    \centering
    \includegraphics[width=.5\textwidth]{Pictures/Kochkurve.png}
    \caption{Kochkurve}
    \label{fig:Kochkurve}
\end{figure}

\subsubsection{Mandelbrotmenge}
Die nach dem Mathematiker Benoît B. Mandelbrot (*20.11.1924; †14.10.2010) bennante Menge ($\mathbb{M}$) beinhaltet jede Zahl $c$ die nicht gegen $\infty$ divergiert für folgende Folge:
\begin{align*}
z_0&=0\\
z_{n+1}&=z^2_n+c\\
\end{align*}
Man fand heraus, dass wenn $|z_n|>|c|$ oder $|z_n| > 2$ gilt, wird die Folge gegen $\infty$ divergieren.\\
Das erstellte Bild, mit ein bisschen Farbe jenachdem dazu, ergibt es ein sehr schönes Gebilde, dass den deutschprachigen Leute an einem 'Apfelmännchen' errinerte, weshalb es so auch genannt wird. Denn es sieht so aus, als ob mehrere Äpfel aufeinander gestapellt sind.\\
Das $\mathbb{M}$ ist selbstähnlich. Man findet das erst gesehen Bild der Menge beim reinzoomen immer wieder, somit ist es eine Art Fraktal. Ebenfalls findet man immer wieder mal verschiedene Julia Mengen mit dem Zugehörigem $c$. Diese sind ebenfalls Fraktale und sehen dazu auch noch für die Norm der Menschheit schön aus. $\mathbb{M}$ besitzt durch die vorgegebene Formel ein chaotisches System.\\
All dies führte dazu, dass es einige YouTube-Viedeos gab, die einen Zoom in die Menge zeigen, welche auch viel Rechenleistung brauchen.
\subsubsection{Buddhabrotmenge}
Schaut man den Namen als Erstes an, wird gemerkt, dass das Wort Buddha vom meditierendem Buddha kommt, den dieser ist in der Menge ersichtlich. Ebenfalls das Wort Brot fällt auf, dies ist eine Andeutung, dass diese Menge etwas mit dem Mandelbrot zu tun hat. \\ Das Abbild entseht indessen mand nochmals das Mandelbrot berechnet, jedoch nur die Punkte, die bei der Mandelbrotberechnung gegen die $\infty$ divergieren. Nun wird auch nicht mehr geschaut nach wievielen Schritten es ins unendliche abdriftet, sondern bei welchen Punkten es nach jeder Iteration landet.\\
Ebenfalls durch das chaotische System von $\mathbb{M}$ und die fraktalen und selbstähnlichen Eigenschaften der Buddhabrotmenge ist ein Zoom ebenfalls intressant.