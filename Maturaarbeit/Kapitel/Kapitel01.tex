\section{Einleitung}
An der Kantonsschule Glarus werden im Schwerpunktfach Anwendung der Mathematik und Physik unter anderem auch die Julia-Mengen und die Mandelbrotmenge angeschaut. Kurz wird ebenfalls das Buddhabrot erwähnt. Man schaut auch einen Zoom an, welcher in die Mandelbrotmenge hineinfokusiert. Es sind jedoch wenige Zooms ins Buddhabrot zu finden, geschweige denn solche, welche gleich tief in das Buddhabrot gehen, wie die beim Mandelbrot.\\ 
Ziel dieser Arbeit ist es, einen Zoom vom Buddhabrot zu berechnen. Es wird nach einer möglichst effizienten Methode für die Berechnung gesucht. Es soll mit rein mathematischen Algorithmen bewerkstelligt werden, jedoch werden schon zum Anfang leichte Hilfen von Programmiertricks benutzt. Diese Arbeit wird mit der Programmiersprache Julia erstellt, einer schnellen und verständlichen Sprache. Es wird jedoch kein Video erstellt, sondern ein Bild. Ein Video ist eine rasche Abfolge von Bildern. Hat man also eine Methode für das Bild, so ist der Schritt zum Video schon erleichtert. Jedoch würde dieser zusätzliche Schritt den Rahmen dieser Arbeit strapazieren.