\section{Einleitung}
An der Kantonsschule Glarus werden im Schwerpunktfach Anwendung der Mathematik und Physik unter anderem die Julia-Mengen und die Mandelbrotmenge thematisiert. Auch das Buddhabrot wird in diesem Zusammenhang erwähnt, wobei auch ein Zoom, welcher in die Mandelbrotmenge hineinfokussiert, angeschaut wird. Es sind jedoch wenige Zooms ins Buddhabrot zu finden, geschweige denn solche, welche gleich tief in das Buddhabrot gehen, wie die beim Mandelbrot.\\ 
Ziel dieser Arbeit ist es, einen Zoom in das Buddhabrot zu berechnen. Es wird nach einer möglichst effizienten Methode für die Berechnung gesucht. Dies soll mit rein mathematischen Algorithmen bewerkstelligt werden, jedoch werden schon zum Anfang leichte Hilfen von Programmiertricks benutzt. Diese Arbeit wird mit der Programmiersprache Julia erstellt, einer schnellen und verständlichen Sprache. Es wird jedoch kein Video erstellt, sondern ein Bild, da ein Video eine rasche Abfolge von Bildern ist. Findet man also eine Methode für das Bild, ist der Schritt zum Video bereits erleichtert. Jedoch würde dieser zusätzliche Schritt den Rahmen dieser Arbeit strapazieren.