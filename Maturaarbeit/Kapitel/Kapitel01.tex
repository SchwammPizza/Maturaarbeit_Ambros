\section{Einleitung}

Die Einleitung ist nicht zu beachten da sie nicht MA gemäss ist, gehen Sie zu den Präliminarien...

Das Unendliche zieht die Menschen schon eh und je an, so ebenfalls mich. Es ist unbeschreiblich, jedoch spielen/rechnen wir damit gerne. Dazu kommt das unsere Technik immer besser und leistungsfähiger wird. Dies spielt sich zu, denn desto mehr Leistung man hat, umso schneller und näher kann man sich der Unendlichkeit angleichen. Ich bin schon seit ich klein bin von komplexeren Mathematischen Vorgängen verwundert, deshalb schaue ich in meiner Freizeit gerne YouTube-Videos über solche. Eines Tages sah ich das Buddha-Brot auf einem Titelbild eins Videos, als ich das dann ansah, verstand ich nichts, es war auf Englisch, so beschäftigte ich mich vorerst nur mit der Mandelbrot-Menge. Erst als ich in der Schule die Fraktalen kennenlernte, verstand ich es besser. Nun mit mehr Erfahrung in der Mathematik und ebenfalls in der Informatik, setzte ich mir zum Ziel ein Zoom reinzurechnen. Jedoch finde ich ein Zoom geht schon, es soll anspruchsvoller sein. So entschloss ich mich die schnellste Methode zu suchen, da das Rendern länger geht als die vom Mandelbrot. Ich werde vorerst versuchen dies rein mathematisch hinzukriegen und nur leichte Hilfe von Programmiertricks nehmen, jedoch würde ich am Schluss es gerne mit einer Methode vergleichen, die auf Optimierung des Codes und vielen Tricks der Informatik beinhaltet, denn dies kann in Kombination das Schnellste sein. Ich werde dies in der Programmiersprache Julia programmieren, eine schnelle und verständlich zu lesende Sprache.
